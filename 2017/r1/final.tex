\documentclass[UTF8]{ctexart}
\title{2017 暑期一轮集训总结与个人分析}
\author{傅宣登}
\begin{document}
\maketitle

\section{一轮集训总结}

今年的暑期一轮集训已接近尾声,感觉比去年好很多。

去年虽然一个暑假都在集训,但是效果很糟糕。
一方面自己不太认真,得过且过,晚上因为写工程不好好休息,导致第二天精神状态极差;
另一方面知识储备有限,赛后也没有补题,还因为写工程耽误了学习算法的时间。
仔细观察去年的 $rating$ 曲线,其实我在开始几场的表现还不错,
如果当时能赛后认真补题、同时查漏补缺学习新算法与数据结构的话,
最后还是有希望保持在 15 名左右的水平。

所以在这一轮开始之前,我就警惕自己不要重蹈去年的覆辙,也很担心会水下去。
不过目前来看,我应该不会再垫底了,感觉自己还算认真地完成了一件事,挺欣慰的。

纵观整个暑期一轮集训,每一周的特点都比较明显。
第一周在蓝名区域波动,也一度绿过。
第二周有点出乎意料,凭借连续几场 $rank 5$ 迅速上分,
$rating$ 涨幅达到了 600。
第三周最后一周感觉有点萎靡,
第一天打了个垫底 $rating$ 一落千丈,
第二天回升了一点后又接着下降,在黄蓝之间波动。

总的来说,我对今年我暑期一轮集训的表现基本满意。
经过统计,我在一轮期间解题超过 70 道、编写 ac 代码超过 165.4 千字节,
是为我搞算法竞赛以来最努力的一个月了。

这三周除了得到了解题的训练之外,还在暑假前集训的基础上
新学习了网络流问题的建模与实现、差分约束系统等新算法,
巩固了 sg 函数的应用等问题。
但是还有很多基础没有学习,这是此次一轮集训的遗憾之处。

作息方面,没有像去年那样不规律,基本按时睡了觉起了床。
也坚持住了拒绝项目的邀请,认真地在搞算法竞赛。

总的来说,今年暑期一轮集训收获不少,但也没有达到最好的效果。

\section{个人分析}

通过对一轮集训中过的题目的分类总结与解体过程中的感受和观察,
总的来说,我比较爱思考数学和几何题,
在数据结构和图论方面表现一般,
一般的动态规划问题可以解决,
在搜索和字符串方面则是很弱。

具体来说,
数据结构方面我掌握了
树状数组、线段树、并查集、Sparse-Table、字典树等基本的数据结构,
但一些高级的数据结构还没有接触过。

图论方面,我能解决
拓扑排序、常见单(多)源最短路问题、
最小生成树、简单的最大流问题以及 LCA 问题和
差分约束系统等。

对于动态规划,
基础的动态规划、树型 DP、状压 DP 我都问题不大,
但在动态规划的优化上想法不多。

搜索和字符串是我的弱项。
搜索基本上只会爆搜,对剪枝也不是很有感觉。
字符串中虽然知道 KMP、AC 自动机和后缀数组的思想,
但没有系统地写过与练过题。

数学是我比较喜欢的一类,但是我常常由于能力不够解不出来。
在数论方面目前我掌握了
(扩展)欧几里得算法、中国剩余定理、线性素数筛和欧拉函数的计算。
组合数学较弱,只会容斥,母函数和 Polya 定理都还不会。
博弈论的话一些比较简单的 Nim 游戏和 SG 函数解决的问题都能解决。
在计算几何上会基本的算法和凸包,但更深入的问题还没有系统研究过。

我从小学四年级开始学习 C 语言和其他一些编程语言,
有接近七年(已除去高中三年)的代码编写经历,在编码经验上略有优势。
因为以前经常写网站和桌面应用,所以我的代码风格偏向工程代码,
有点固执地秉持注重复用、高内聚、低耦合等一些工程思想(虽然在大部分时候并不适合算法竞赛),
在写超大模拟题上有一点优势。

对我来说,读题一般不是问题。

\end{document}
